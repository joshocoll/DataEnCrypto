\documentclass{article}
\author{Nishedcob et al.}
\title{Una Guia de como usar DataEnCrypto}
\begin{document}
  \maketitle
  \tableofcontents
  \section{¿Que es DataEnCrypto?}
    DataEnCrypto es un proyecto de codigo abierto que es un expirimento de varias formas de encriptar informacion. El idea es que uno puede ver el codigo y entender como cada metodo funciona y como es su implimentacion en Java. Ahora es un proyecto del ciclo, pero empezando Febrero 2014 dejara de ser esto y tal vez lo abrire para que todo el mundo, para los que quieren, pueden ayudar en su desarollo. Vivimos en una epoca peligroso con la NSA, GCHQ y otros agencias de espionaje digital queriendo accesso a todos nuestros datos y communicaciones personales -- los que quieren privacidad necesitan proteger sus datos de alguna forma -- no poner los en linea, o usar las varias metodos de encriptacion que hay para esconderlos. Yo, como programado, como estudiante y como joven, tengo una resposibilidad de ver que todos tengan accesso a las harramientas que necesitan para proteger su propio informacion y que estas harramientas sean de una forma abierta para que, si tu quieres, puedes revisar el codigo abajo y verificar que no hay puertas traseras hecho por las mismas agencias o personas de cual queremos olcultar nuestros datos.

    Uno se puede leer mas, y informacion mas actualizada en nuestro wiki: https://github.com/nishedcob/DataEnCrypto/wiki
\end{document}
